% Permission is granted to copy, distribute and/or modify this document
% under the terms of the GNU Free Documentation License, Version 1.3
% or any later version published by the Free Software Foundation;
% with no Invariant Sections, no Front-Cover Texts, and no Back-Cover Texts.
% A copy of the license is included in the section entitled "GNU
% Free Documentation License".
%
% Written (C) 2012 Chiyuan Zhang

\chapter{Multiclass learning}

In this chapter we describe multiclass learning algorithms available in the
\shogun{} toolbox. 
Multiclass learning refers to the problem with the output space
$\mathcal{Y}=\{1,\ldots,K\}$, where $K>2$.  Most of real world machine learning
classification problems are naturally multiclass.   Typical examples include
document categorization, image classification, hand-written digit recognition,
etc.  

Generally, no assumption of any specific structure for the set $\mathcal{Y}$
are made in multiclass learning.  When priori knowledge are available for a
rich structure of $\mathcal{Y}$, \emph{structured-output learning} algorithms
are usually used instead.

\section{Reduction to Binary Problems}

Since binary classification problems are one of the most thoroughly studied
problems in machine learning, it is very appealing to consider reducing
multiclass problems to binary ones. Then many advanced learning and
optimization techniques as well as generalization bound analysis for binary
classification can be utilized.

In \shogun{}, the strategies of reducing a multiclass problem to binary
classification problems are described by an instance of
\shogunclass{CMulticlassStrategy}. A multiclass strategy describes
\begin{enumerate}
\item How to train the multiclass machine as a number of binary machines?
	\begin{itemize}
		\item How many binary machines are needed?
		\item For each binary machine, what subset of the training samples are
			used, and how are they colored\footnote{In multiclass problems, we
				use \emph{coloring} to refer partitioning the classes into two
				groups: $+1$ and $-1$, or black and white, or any other meaningful
				names.}?
	\end{itemize}
\item How to combine the prediction results of binary machines into the final
	multiclass prediction?
\end{enumerate}

The user can derive from the virtual class \shogunclass{CMulticlassStrategy} to
implement a customized multiclass strategy. But usually the built-in strategies
are enough for general problems. We will describe the built-in \emph{One-vs-Rest},
\emph{One-vs-One} and \emph{Error-Correcting Output Codes} strategies in the
following subsections.
